\documentclass{article}
%\usepackage{fullpage}
\usepackage{fancyhdr}
\usepackage[english,francais]{babel}
\usepackage[T1]{fontenc}
\usepackage[utf8]{inputenc}
\usepackage[pdftex]{graphicx}
\usepackage{subfig}

%\renewcommand{\baselinestretch}{2}
\author{Florent \textsc{Guiotte} et Frédéric \textsc{Becker}}
\title{Filtrage}
\pagestyle{fancy}

\begin{document}
\maketitle
\tableofcontents

\section{Questions}
\subsection{Ouverture et fermeture}

\subsection{Gradient morphologique}

\subsection{Laplacien}

\subsection{Contours}
Pour detecter les contours de la première image il suffit d'appliquer le filtre Laplacien directement, puis de detecter
les passages à zeros sur le resultat.

Pour la deuxieme image, elle est bruitee par du bruit de type <<sel et poivre>>, on applique un pretraitement. La première etape est de faire une
ouverture afin de supprimer le buit sel, puis une fermeture afin de supprimer le bruit poivre. Pour finir on fait les
mêmes etapes que la
premiere image.

Pour la troisieme image, elle est bruitee avec du bruit blanc. Un filtre passe bas permet de l'attenuer, nous avons
utilise le filtre moyenneur (Le resultat n'est cependant pas satisfaisant, la difference entre le bruit et les contours
est trop faible). Pour finir avec les etapes de la premiere image.
\subsection{Combinaison de filtres}

H1 et h2 sont deux filtres gaussien d'amplitude differentes. La soustraction des deux premières images donnent les mêmes
contour que le seul filtre 3. 

\end{document}
